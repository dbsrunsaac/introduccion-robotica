\chapter{Introducción}

Desde el origen poético de la robótica ligado a las sirvientes de Hefesto en la obra Iliada de Homero hasta las diversas aplicaciones en la actualidad se refleja el hecho de que los procesos y aplicaciones más complejas surgen de la capacidad del ser humano para soñar e imaginar, es ahí que dentro de tal abanico de opciones surgen aplicaciones como la biomédica que implica la unión de 2 ramas de la ciencia	entre las cuales se encuentra la robótica y la medicina para el tratamiento de enfermedades y cuidados posteriores que agregan precisión y altas tasas de efectividad en procedimientos delicados.

Actualmente esta evolución aunque aún en surgimiento se destaca en conjunto con el desarrollo de nuevas tecnologías como la inteligencia artificial y la red 6G lo cual nos da a entender que su evolución es un esfuerzo en conjunto como el propio nacimiento de la rama perse y en conjunto a la necesidad de mejorar los procedimientos como se hizo en su momento con los robots industriales, en Perú y el resto de países la métrica asociada a la cantidad de robots en medicina  tiene un constante crecimiento el cual se refleja en el primer prototipo equipado en 2021 para cirugias menores como la laparoscopia.

El presente trabajo enfatiza la integración de la robótica en la medicina y la revolución que esta tecnología agrega en los procedimientos y cuidado de la salud, ofreciendo herramientas precisas y eficientes que amplían las capacidades humanas como lo es requerido en la cirugía asistida, integrando para ello sistemas como el Da Vinci y Robodoc que permiten intervenciones mínimamente invasivas con mayor precisión y control remoto, reduciendo riesgos y tiempos de recuperación, así mismo en el área de los robots vestidos, las prótesis mioeléctricas, evolucionadas desde la década de 1960 se utilizan para restaurar funciones en amputados combinando estética, fuerza y versatilidad mediante control EMG, siendo uno de los exoesqueletos destacados el HAL que potencian la movilidad en rehabilitación, detectando bioseñales para multiplicar la fuerza del usuario y facilitar terapias activas o pasivas.

Estos avances, impulsados por instituciones como NASA y DARPA, no solo mejoran la calidad de vida, sino que promueven una colaboración entre ingeniería y ciencias de la salud, abriendo caminos hacia procedimientos autónomos y personalizados.
