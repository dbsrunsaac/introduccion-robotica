\chapter{Desarrollo del tema}

\subsubsection{Manipuladores montados sobre sillas de ruedas}
Los manipuladores robóticos montados sobre sillas de ruedas están diseñados para aumentar la independencia y funcionalidad de personas con limitaciones motrices, permitiéndoles manipular objetos y realizar tareas cotidianas con mayor autonomía. Un ejemplo representativo es el robot MANUS, que se controla por joystick complementado con un teclado para facilitar la interacción con el entorno próximo. Este tipo de sistemas ha demostrado su utilidad tanto en el ámbito doméstico como en el laboral, proporcionando asistencia efectiva en la manipulación de objetos y mejorando la calidad de vida del usuario. Sin embargo, presentan ciertas limitaciones, como la necesidad de alta concentración, incomodidad cuando no están en uso y restricciones en la precisión de movimientos. Además, por su peso y tamaño, pueden afectar la maniobrabilidad de la silla de ruedas en espacios reducidos, limitando su practicidad en algunos contextos

El robot RAPTOR ejemplifica una versión más sencilla y económica, con menor número de grados de libertad y una interfaz básica, llegando al mercado estadounidense tras su aprobación por la FDA. Pese a estas características, al igual que otros manipuladores, es efectivo para usuarios con movilidad disminuida hasta en un 80\%, recuperando parte importante de la capacidad manipulativa. La evolución de estos manipuladores busca superar sus limitaciones actuales para ofrecer soluciones más ergonómicas, versátiles y cómodas, integrando tecnologías que reduzcan la carga física y mental del usuario, aumentando así su autonomía.


\subsubsection{Sistemas móviles autónomos}

Los sistemas móviles autónomos en robótica médica representan una innovadora solución para apoyar el traslado y asistencia dentro de entornos hospitalarios y de cuidado. Estos robots pueden desplazarse sin intervención directa humana, realizando tareas como transporte de materiales, monitoreo de pacientes o asistencia en procedimientos clínicos. Su autonomía y capacidad de navegación en espacios complejos permiten optimizar recursos, disminuir el riesgo de errores humanos y mejorar la eficiencia operativa de los centros de salud. Equipados con sensores y algoritmos de navegación avanzada, estos sistemas pueden detectar obstáculos, adaptar rutas y colaborar dentro del entorno hospitalario sin interferir con otras tecnologías médicas [opensurg2013robotica].

Además, los sistemas móviles autónomos están siendo diseñados para trabajar en coordinación con el personal sanitario y otros dispositivos robóticos, facilitando una asistencia integral y segura. La incorporación de inteligencia artificial y capacidades de aprendizaje aumenta su adaptabilidad y autonomía, posibilitando su uso en contextos cada vez más desafiantes y variados. Esto abre un panorama prometedor para la expansión del uso de la robótica en la atención médica, no solo en cirugía o rehabilitación, sino también en la gestión cotidiana y el soporte logístico hospitalario [opensurg2013robotica].

\subsubsection{Sillas de rueda con guiado autónomo}

Las sillas de ruedas con guiado autónomo incorporan sistemas inteligentes de navegación que facilitan la movilidad de personas con discapacidad motriz, incrementando su independencia. Estas sillas detectan obstáculos y planifican rutas seguras mediante sensores, cámaras y tecnologías de mapeo que les permiten adaptarse a distintos entornos, tanto interiores como exteriores. Así, los usuarios pueden desplazarse con menor esfuerzo y mayor seguridad, reduciendo la dependencia de acompañantes o personal asistencial. El guiado automático ofrece una experiencia más fluida, especialmente útil en espacios complejos o desconocidos, donde la asistencia tradicional puede ser limitada [opensurg2013robotica].

Además, algunas sillas incorporan sistemas de control multimodal que combinan comandos tradicionales (joystick) con asistencia autónoma para facilitar maniobras, optimizar trayectorias y evitar colisiones. Esto permite que el usuario conserve el control durante el desplazamiento, pero con soporte inteligente que mejora la seguridad y confort. El desarrollo continuo en esta área se orienta a mejorar la integración con otros dispositivos médicos y entornos inteligentes, aumentando la autonomía de los usuarios y abriendo nuevos horizontes para la movilidad asistida [opensurg2013robotica].

\subsubsection{Robots asistenciales para la navegación}

Los robots asistenciales para la navegación están diseñados para ayudar a personas con discapacidad visual o motriz a moverse de manera segura e independiente en distintos entornos. Estos robots incorporan tecnologías avanzadas de sensores, cámaras, mapas y algoritmos de inteligencia artificial para guiar y asistir al usuario, evitando obstáculos y ofreciendo indicaciones en tiempo real. Su función es proporcionar orientación precisa y soporte activo durante la navegación, mejorando significativamente la autonomía y seguridad, en especial en espacios complejos, congestionados o poco familiares para los usuarios [opensurg2013robotica].

Además, estos robots suelen integrarse con sistemas de comunicación accesibles que permiten ajustes personalizados según el nivel de discapacidad del usuario, facilitando así un apoyo adaptativo y efectivo. El avance en estas tecnologías está permitiendo el diseño de dispositivos cada vez más compactos, intuitivos y eficientes, con potencial para transformar la experiencia diaria de muchas personas con necesidades especiales, promoviendo la inclusión y la autonomía en diferentes contextos sociales y laborales [opensurg2013robotica].
