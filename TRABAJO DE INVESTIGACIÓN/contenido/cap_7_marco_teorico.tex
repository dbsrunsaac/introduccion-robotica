\chapter{Marco Teórico}

La robótica en el ámbito medico no es un caso reciente, este viene acumulandose en la última decada a travez del crecimiento de investigación y desarrollo en ingeniería, informática y ciencias de la salud. 

Los primeros en desarrollar este modelo de robots asistentes se encargo de hacerlo la NASA (National Aeronautics and Space Administration) y la DARPA (Defense Advanced Research Project Administration), con su unico objetivo de reemplazar a un cirujano en situaciones complejas, como los programas espaciales, en el campo de batalla y en zonas de catástrofes. En estos primeros modelos la separación del paciente del cirujano y del resto del equipo es el elemento clave, es decir, hablamos de la telecirugía \cite{capitan_cirugia_robotica}.

\subsection{Cirugía Laparoscópica (Década de 1980)}
La cirugía laparoscópica es una técnica quirúrgica moderna que permite operar dentro del abdomen o la pelvis sin necesidad de hacer grandes incisiones en la piel.

En 1987 Mouret realizó la primera colecistectomía laparoscópica, marcando un hito en la historia de la CMI (Cirugia minimamente invasiva). Desde entonces el abordaje laparoscópico y mínimamente invasivo se ha consolidado en todas las disciplinas quirúrgicas. \cite{capitan_cirugia_robotica}

En 1995 Frederick Moll y Robert Younge fundaron la empresa Intuitive Surgical Inc (Sunnyvale, California, USA) desarrollando un proyecto completo de robótica quirúrgica  con su primer prototipo denominado Lenny. \cite{camarillo2004robotic}

En 1997 inició sus experiencias con el segundo prototipo denominado MONA, precursor del da Vinci, que utilizaba el sistema maestro-esclavo con una consola de mandos y brazos independientes colocados mediante laparoscopia en el abdomen del paciente. 

Ya que se tenia este sistema implementado en 1997 se acoplo por primera vez 3 brazos de asistencia quirurgicos, lo cual le sirvio al Belga James Himpens para hacer la primera colecistectomia asistida por un robo. 

\subsection{Puma 560 (1985)}
En 1985, Yik S. Kwoh realiza la primera cirugía asistida por robot para la realización de una biopsia cerebral (7) con el denominado PUMA 560 (Programmable Universal Manipulation Arm, Unimation, Stanford, California, USA). Posteriormente James Drake y cols. lo 
utilizan para la resección de astrocitomas talámicos en niños. \cite{drake1991computer} 

\begin{figure}[h]
    \centering
    \includegraphics[width=0.5\linewidth]{1.png}
    \caption{Robot Puma 560.}
    \label{fig:1}
    %\begin{flushleft}
    %    \textit{Nota.} Adaptado de *[Título de la fuente]*, por [Autor/a o Institución], %[Año], [Editorial o enlace]. 
    %\end{flushleft}
\end{figure}

\subsection{Rodoboc (1992)}

A mediados de la década de 1980 la tecnología robótica quirúrgica más avanzada en uso hoy en día, fue concebida por el difunto veterinario de Sacramento, Dr. Howard “Hap” Paul, y el Dr. William Bargar, cirujano ortopédico. Ambos cirujanos se propusieron encontrar una herramienta que pudiera tallar una cavidad en un fémur que se ajustara con precisión a la forma de un implante de cadera artificial. 

Una versión anterior del sistema ROBODOC se utilizó por primera vez en 26 pacientes del Dr. Paul, todos perros domésticos con lesiones de cadera. Todas las cirugías caninas fueron exitosas.El siguiente paso fue realizar cirugías en pacientes humanos, y el 7 de 
noviembre de 1992, en el Hospital General Sutter de Sacramento, California, se realizó la primera cirugía de reemplazo de cadera en humanos con la herramienta ROBODOC. \cite{pransky1997robodoc}. 

En octubre de 1993, tras la finalización exitosa del estudio de viabilidad con diez pacientes, la FDA autorizó un programa ampliado de hasta 300 operaciones (150 con ROBODOC y 150 en un grupo de control manual). 

\subsection{Primeras Prótesis Mioeléctricas (Décadas de 1960-1980)}

Las prótesis controladas por señales mioeléctricas comenzaron a desarrollarse en Rusia hacia 1960. Este tipo de dispositivo emplea los pequeños potenciales generados durante la contracción de los músculos del muñón, los cuales son transmitidos y amplificados para producir el movimiento. En un inicio, solo se utilizaban en personas con amputación de antebrazo y alcanzaban una fuerza de prensión aproximada de dos kilogramos.

\begin{figure}[h]
    \centering
    \includegraphics[width=0.5\linewidth]{2.png}
    \caption{Configuración básica de una prótesis mioeléctrica}
    \label{fig:3}
    %\begin{flushleft}
    %    \textit{Nota.} Adaptado de *[Título de la fuente]*, por [Autor/a o Institución], %[Año], [Editorial o enlace]. 
    %\end{flushleft}
\end{figure}

Las prótesis mioeléctricas son prótesis eléctricas controladas por medio de un poder externo mioeléctrico, estas prótesis son hoy en día el tipo de miembro artificial con más alto grado de rehabilitación. Sintetizan el mejor aspecto estético, tienen gran fuerza y velocidad de prensión, así como muchas posibilidades de combinación y ampliación. 

El control mioeléctrico es probablemente el esquema de control más popular. Se basa en el concepto de que siempre que un músculo en el cuerpo se contrae o se flexiona, se produce una pequeña señal eléctrica (EMG) que es creada por la interacción química en el cuerpo. El uso de sensores llamados electrodos que entran en contacto con la superficie de la piel permite registrar la señal EMG. Una vez registrada, esta señal se amplifica y es procesada después por un controlador que conmuta los motores encendiéndolos y apagándolos en la mano, la muñeca o el codo para producir movimiento y funcionalidad. 


\subsection{Investigación en Cibernética y Biónica (Mitad del siglo XX)}


