\begin{resumen}
	
\noindent

Los hitos históricos de la robótica datan desde su inicio en el colectivo humano como entidades artificiales que puede ejecutar acciones de forma autónoma complementando así el trabajo que el ser humano ha realizar es en este proceso de transformación que esta rama multidisciplinaria poco a poco va cambiando el rumbo de la medicina como área independiente integrando herramientas cuyo inicio fue marcado por la NASA y DARPA con robots diseñados para telecirugía en entornos extremos, como programas espaciales o zonas de catástrofe, sistemas pioneros como Puma 560 (1985) y Robodoc (1992) marcaron hitos en biopsias cerebrales y reemplazos articulares mientras que en aplicaciones que requieren de precisión el robot Da Vinci consolida técnicas que eliminan las limitantes humanas ofreciendo cortes milimétricos reduciendo el área afectada y tiempo de recuperación posoperatoria, mientras que en el otro extremo se pueden observar a los robots vestibles u exoesqueletos como elementos de soporte para restaurar la movilidad en personas con discapacidad.

\end{resumen}